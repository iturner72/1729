\documentclass{article}
\usepackage{graphicx}
\graphicspath{ {./../../pix/} }
\usepackage{wrapfig}
\usepackage{xcolor}
\definecolor{light-gray}{gray}{0.95}
\newcommand{\code}[1]{\colorbox{light-gray}{\texttt{#1}}}
\usepackage[utf8]{inputenc}
\usepackage[english]{babel}
\usepackage[]{amsthm}
\usepackage[]{amssymb}

\title{1729 Elixir Project}
\author{Ian Turner}

\begin{document}
\maketitle

\section*{Elixir 1729 Project}
This Project is worth 100 dollars in BTC

\section*{Intro to Elixir}
In interactive mode I entered $\code{"hello" " world"}$, and the terminal
returned: "hello world". I have quickly, I think, learned how the string
syntax works.

\subsection*{Running The Scripts}
After getting familiar with the basics of the lanquage you may want to try
writing simple programs. This can be accomplished by putting the following
Elixir code into a file:\\
\code{IO.puts("Hello world from Elixir")}\\
Save it as \code{simple.exs} and execute it with \code{elixir}:\\
\code{elixir simple.exs}

\section*{Elixir School}
Elixir School seems to have a great overview of elixir.

\subsection*{Basic Data Types}
\textbf{Integers}: Support for binary, octal, and hexadecimal numbers.\\
\code{0b0110} $\rightarrow 6$\\
\code{0o644} $\rightarrow 420$\\
\code{0x1F} $\rightarrow 31$\\
\textbf{Floats}: In Elixer, floating point numbers require a decimal after at
least one digit: they have 64-bit double precision and support \code{e} for
exponent values:\\
\textbf{Booleans}:\\
\textbf{Atoms}:\\
\textbf{Strings}:



\end{document}
